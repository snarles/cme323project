\documentclass{article} % For LaTeX2e
\usepackage{nips13submit_e,times}
\usepackage{hyperref}
\usepackage{url}
\usepackage{amssymb, amsmath}
\usepackage{epsfig}
\usepackage{array}
\usepackage{ifthen}
\usepackage{color}
\usepackage{fancyhdr}
\usepackage{graphicx}
\usepackage{algorithm}
\usepackage{algpseudocode}
\usepackage{mdframed}
\usepackage{amsthm}
%\documentstyle[nips13submit_09,times,art10]{article} % For LaTeX 2.09
\newtheorem{theorem}{Theorem}[section]
\newtheorem{lemma}[theorem]{Lemma}
\newtheorem{corollary}[theorem]{Corollary}

\title{All-Pairs Shortest Paths in Spark}


\author{
Charles Y.~Zheng and Jinshu Wang\\
Department of Statistics\\
Stanford University\\
Stanford, CA 94305 \\
\texttt{\{snarles, jinshuw\}@stanford.edu} \\
\and
\textbf{Arzav ~Jain} \\
Department of Computer Science\\
Stanford University\\
Stanford, CA 94305 \\
\texttt{arzavj@stanford.edu} \\
}

% The \author macro works with any number of authors. There are two commands
% used to separate the names and addresses of multiple authors: \And and \AND.
%
% Using \And between authors leaves it to \LaTeX{} to determine where to break
% the lines. Using \AND forces a linebreak at that point. So, if \LaTeX{}
% puts 3 of 4 authors names on the first line, and the last on the second
% line, try using \AND instead of \And before the third author name.

\newcommand{\fix}{\marginpar{FIX}}
\newcommand{\new}{\marginpar{NEW}}

\nipsfinalcopy % Uncomment for camera-ready version

\begin{document}


\maketitle

\begin{abstract}
We propose an algorithm for the All-Pairs-Shortest-Paths (APSP)
problem suitable for implementation in Spark, and analyze its
performance.  We begin by considering distributed Floyd-Warshall, as
proposed by Kumar and Singh (1991).  Distributed Floyd-Warshall has
asymptotically optimal scaling and can be implemented in Spark by
using BlockMatrix to represent the APSP distance matrix.  However, we
observe that its implementation in Spark suffers from poor performance
for medium-sized problems due the large number of global updates of
the APSP distance matrix required for the algorithm.  Since the
lineage of the algorithm grows with the number of vertices $n$, it
becomes necessary to use a proportional number of checkpoints which
further impacts the efficiency of the algorithm. This motivates the
consideration of an algorithm for APSP which requires fewer global
update steps.  We adapt an approach by Solomonik et al. (2013) based
on the ``divide and conquer'' algorithm for APSP.  Our algorithm
reduces the number of global updates by a factor of $b$, where the
block size $b$ determines the amount of computation done in each
iteration.  By adjusting the block size $b$ we obtain a favorable
tradeoff between checkpointing costs and computation cost per
iteration, resulting in far improved performance compared to
Distributed Floyd-Warshall.
\end{abstract}

\section{Summary}

For the convenience of the grader we present an overview of our
approach and our results.  The rest of the paper gives a detailed
explanation of the results in this section.

\subsection{Problem Specification}

Let $G = (V, E)$ be a graph with $n$ vertices.  Assume the input is
given in the form of the adjacency matrix $A$ of the graph stored as a
{\tt BlockMatrix} with equally sized square blocks.  Specifically
define the adjacency matrix $A$ as a square matrix with dimension $n =
|V|$, and entries
\[
A_{ij} = 
\begin{cases}
w_{i, j} &\text{ if } (i \to j) \in E\\
0 &\text{ if } i = j\\
\infty &\text{ if } (i \to j) \notin E
\end{cases}
\]

Let $b$ be the size of the block, and let $n = b\ell$ so that $\ell^2$
is the number of blocks.  Write
\[
A = \begin{pmatrix}
A^{11} & A^{12} & \cdots & A^{1\ell}\\
A^{21} & A^{22} & \cdots & A^{2\ell}\\
\vdots & \vdots & \ddots & \ddots\\
A^{\ell 1} & A^{\ell 2} & \cdots & A^{\ell \ell}
\end{pmatrix}
\]
so that $A^{ij}$ is the $(i,j)$th block in the {\tt BlockMatrix}.

The output is given by the APSP distance matrix $S$, where
\[
S_{ij} = 
\begin{cases}
\text{weight of shortest path} &\text{ if there exists a path } i \to j\\
0 &\text{ if } i = j\\
\infty &\text{ if there is no path } i \to j
\end{cases}
\]
Let $S$ be stored as a {\tt BlockMatrix} with the same dimensions and
block sizes as $A$, so that
\[
S = \begin{pmatrix}
S^{11} & S^{12} & \cdots & S^{1\ell}\\
S^{21} & S^{22} & \cdots & S^{2\ell}\\
\vdots & \vdots & \ddots & \vdots\\
S^{\ell 1} & S^{\ell 2} & \cdots & S^{\ell \ell}
\end{pmatrix}
\]

Let $p$ be the number of workers.  Let $M$ be the memory of ech
worker.  Suppose each worker holds $K$ contiguous blocks.  It must be
the case that $K < M/b^2$.  In fact, $K$ is even smaller because each
worker will have to hold additional data in memory.

\subsection{Scaling}

We consider the scaling $n \to \infty$ and $p \to \infty$.  We do
\emph{not} assume a sparse graph $G$, so the number of edges can scale
as $E \sim n^2$.  However $b$ must be constant since we assume each
block must fit in memory.

\subsection{Notation}

Given an $n \times k$ matrix $A$ and a $k \times m$ matrix $B$, define
the \emph{min-plus} product $C = A \otimes B$ by
\[
C_{i,j} = \min_{l = 1}^k A_{il} + B_{lj}
\]
for $i = 1,\hdots, n$ and $j = 1,\hdots, m$.

Define $\text{APSP}(A)$ as the all-pairs-shortest-distance matrix for
adjacency matrix $A$.  For example, $\text{APSP}(A)$ is obtained by
running the Floyd-Warshall algorithm on $A$.

\subsection{Algorithm}

The algorithm consists of an \emph{outer loop} with $\ell = n/b$ iterations.
Each iteration culminates in the global update of the {\tt
  BlockMatrix} $S$ containing the intermediate values of the APSP
distance matrix.  Each outer loop iteration involves the execution of
three distributed subroutines in sequence, called the A-step, the
B-step and C-step.  In addition, after every $q$ iterations, the {\tt
  BlockMatrix} $S$ is checkpointed.

We first give an shorthand description of the algorithm without
explicitly specifying the Spark operations used in each step or what
data needs to be communicated at each step.  In the analysis, we
expand each step to describe the specific Spark operations needed,
including the broadcasts, joins, etc. needed to transfer the necessary
data across workers.  Do note that $S^{(0)}, S^{(1)},\hdots,
S^{(\ell)}$ refer to the sequence of {\tt BlockMatrix} objects storing
the results of each iteration.

\begin{algorithm}[H]
\caption{Distributed Block APSP (shorthand)}
\begin{algorithmic}
\Function{BlockAPSP}{Adjacency matrix $A$ given as a {\tt BlockMatrix} with $\ell$ row blocks and $\ell$ column blocks}
  \State $S^{(0)} \leftarrow A$
  \For{$k = 1,\hdots, \ell $}
    \State [A-step]
    \State $S^{kk(k)} \leftarrow \text{APSP}(S^{kk(k-1)})$
    \State [B-step]
    \For{$i =1,\hdots, \ell,\ j = 1,\hdots, \ell$} \emph{in parallel}
      \If{$i = k$ and $j \neq k$}
        \State $S^{kj(k)} \leftarrow S^{kk(k)} \otimes S^{kj(k-1)}$ 
      \EndIf
      \If{$i \neq k$ and $j = k$}
        \State $S^{ik(k)} \leftarrow S^{ik(k)} \otimes S^{kk(k)}$
      \EndIf
    \EndFor
    \State [C-step]
    \For{$i = 1,\hdots, \ell,\ j = 1,\hdots, \ell$} \emph{in parallel}
      \If{$i \neq k$ and $j \neq k$}
        \State $S^{ij(k)} \leftarrow S^{ik(k)} \otimes S^{kj(k)}$ 
      \EndIf
    \EndFor
    \State [D-step]
    \If{$k \equiv 0 \mod q$}
      \State Checkpoint $S^{(k)}$
    \EndIf
  \EndFor
  \State Return $S = S^{(n/b)}$, the APSP matrix in {\tt BlockMatrix} form
\EndFunction
\end{algorithmic}
\end{algorithm}

\subsection{Optimality}

The single-core cost of Floyd-Warshall, the best known single-core
algorithm for APSP, is $O(n^3)$.  A perfectly distributed form of
Floyd-Warshall therefore has a total runtime of $O(n^3/p)$, in the
asymptotic regime $n \to \infty$ and $p = O(n)$.  Our algorithm
achieves the same asymptotic runtime of
\[
O\left(\frac{n^3}{p} + \frac{n}{b} + \right)
\]
for details see section 3.

One can also consider the \emph{communication cost} scaling in terms
of the amount of data tranferred over the network.  The paper by
Solomonik et al. (2013) derived a theoretical lower bound on the
communication cost of APSP as $\Omega(\frac{n^2}{p^{2/3}})$ words.  In
comparison, our algorithm communicates a total of $O(n^2\sqrt{p} + \frac{n}{b}\sqrt{p})$
words, which is worse by a power of $p$.

\subsection{Communication Cost and Type}

We analyze the \emph{bandwidth} (total words sent) and the type of
communication in each step of the algorithm.  The A-step involves a
one-to-one communication of a matrix of size $b \times b$ from a
worker to the driver, involving a bandwidth of $O(b^2)$ words.  The
B-step involves a one-to-all broadcast of a matrix of size $b\times b$
from the driver to $\sqrt{p}$ workers, hence a bandwidth of $O(b^2
\sqrt{p})$ words.  The C-step involves an all-to-all communication (a
map-side join) where each worker recieves two $n/\sqrt{p} \times b$
matrices, and therefore entails a bandwidth of $O(nb\sqrt{p})$.
Therefore the per-iteration bandwidth is $O((1 + \sqrt{p})b^2 +
\frac{nb}{\sqrt{p}})$.  The total bandwidth for the algorithm is
$O(n^2\sqrt{p}b + n(1+\sqrt{p})b)$.  See section 3 for the derivation
of the bandwidth per step.

\section{Background}

\section{Analysis}

In each step we give the \emph{computational cost} (computation done
on each worker), the \emph{bandwidth} (number of words sent) and the
total \emph{runtime} of the entire step.  We give an exact
(non-asymptotic) analysis given the following assumptions:

\begin{enumerate}
\item The time it takes to run Floyd-Warshall on a local matrix of size $b \times b$ is given by $\kappa_F b^3$
\item The time it takes to locally perform min-plus multiplication
  on matrices of size $a \times b$ and $b \times c$ be given by $\kappa_M ab^2c$
\item Separate the cost of communication and computation, so that
  sending messages and receiving messages is not included in the
  computational cost.
\item Time taken to send one message consisting of $m$ words from one
  machine (whether a worker or driver) to another machine is
\[
T(m) = \kappa_L + \kappa_T m
\]
where $\kappa_L$ is a latency constant describing the time it takes
for data to travel through the network and $\kappa_T$ is the
transmission time per word.
\item Each machine has a probability $\epsilon$ of instantaneously
  failing.  The machine is instantly replaced, but all data in memory
  and disk is lost.
\item Let the time it takes to write to disk be given by $r\kappa_C
  m$, where $\kappa_C$ is a constant for the disk write time and $m$
  is the amount of data each worker has to write.  This is assuming
  the data is already on disk.
\item Assume no time cost for deleting data from disk.
\item We assume that the initial data $A$ is stored on fault-tolerant
  backup, otherwise there is no guarantee that the job can complete.
  We assume a cost of $\kappa_B n^2$ to restore the cluster to the
  initial state from backup.
\end{enumerate}

Note in particular that assumption 4 assumes a \emph{uniform} network
topology so the transmission rate from any worker to any other worker
is equal.  This is in contrast to the \emph{grid} or \emph{hypercube}
topologies considered in most of the existing literature on
distributed APSP.

Assumptions 5-8 deal with fault-tolerance.  In the following we will
start by giving an analysis of \emph{fault-free iterations}, and then
as we analyze the global cost, we incorporate the extra cost from
faulty iterations.

\subsection{A-step}

\subsection{B-step}

\subsection{C-step}



\bibliographystyle{abbrv}
\bibliography{isomap}



\end{document}
